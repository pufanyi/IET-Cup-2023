这道题的难点在于,Jianheng 可以反复让 Oscar 重复经过一个格子很多次,导致很多猜想事实上是不正确的。

首先,我们可以看到,Jianheng 其实每次最多只需要放一堵墙。因为 Oscar 每次只会移动一格,Jianheng 只需要决定他是否应该在 Oscar 正下方放墙就可以了。

然后一个结论就是如果 Jianheng 选择不在棋子下面放墙,那么 Oscar 的最优策略一定是往下走。因为他如果往左或者右走,那么 Jianheng 完全可以将这行的其他格子下面全部放上墙,这样 Oscar 只能折返,

然后我们需要发现一个结论:对于每一行,在两人都是绝对聪明的情况下,Jianheng 放的墙一定是连续的。因为假设 Jianheng 想让 Oscar 从 $(x, y)$ 走到 $(x, y+1)$,并且 Oscar 现在在 $(x_0, y)$。那么他可以通过一直 Oscar 下面那个格子的方式,逼着 Oscar 走到 $(x, y)$。这时候由于 Oscar 是聪明的,他不会选择继续横着走。因为如果这样,Jianheng 就可以堵住这一行除了 $(x, y)$ 之外的其他所有格子,Oscar 就必须折返,反而多吃了很多罚分。

假设现在 Oscar 在 $(i, l)$,而 $l\sim r$ 这一段是有墙的。那么现在 Oscar 有两种方式:一种是一直向右走到 $(i, r+1)$,一种是向左走到 $(i, l-1)$。在 $(i, r)$ 时同理。

这时候我们就能列出 DP 状态了:$f_{i, l, r, 0/1}$ 表示目前是 Jianheng 操作,棋子现在在第 $i$ 行,现在 Jianheng 在第 $i$ 行放的墙的是 $l\sim r$ 的一段,最后一个维度如果是 $0$ 表示棋子现在在 $(i, l-1)$,$1$ 表示棋子在 $(i, r+1)$。转移的时候 Jianheng 先选择放还是不放墙。如果放墙的话,那么 Oscar 可以选择向左或者向右走;如果不放墙,那么 Oscar 必须选择向下走。
